\documentclass{ctexart}

% Language setting
% Replace `english' with e.g. `spanish' to change the document language
\usepackage[english]{babel}

% Set page size and margins
% Replace `letterpaper' with `a4paper' for UK/EU standard size
\usepackage[a4paper,top=2cm,bottom=2cm,left=3cm,right=3cm,marginparwidth=1.75cm]{geometry}

% Useful packages
\usepackage{amsmath}
\usepackage{amssymb}
\usepackage{graphicx}
\usepackage[colorlinks=true, allcolors=blue]{hyperref}

\title{函数专题}
\author{Siyuan Zhou}

\begin{document}
\maketitle

\section*{周期性(易和三角函数结合考察)}

\subsection*{定义:}
\textbf{周期函数、周期}:对于定义在$\mathbb{D}$上的函数$f(x)$,若存在非零常数$T$,使得$\forall x \in \mathbb{D}: f(x+T) = f(x)$,则称$f(x)$为周期函数,$T$称为$f(x)$的一个周期。\\
\par
\textbf{最小正周期}:对于周期函数$f(x)$,若在其所有周期中,存在最小正数$T_0$,则称$T_0$为$f(x)$的最小正周期。

\subsection*{例题:}
\begin{enumerate}
  \item $f(x)=\sin(x^2)$ 是周期函数吗?若是,给出其周期;若不是,证明之。
  \vspace{20mm}
  \item $f(x)=\sin(x) + \sin(\sqrt{2}x)$ 是周期函数吗?若是,给出其周期;若不是,证明之。\\
    推广:$f(x)=\sin(\omega_1x) + \sin(\omega_2x)$ 为周期函数 $(\omega_1,\omega_2 \neq 0) \Leftrightarrow \frac{\omega_1}{\omega_2} \in \mathbb{Q}$
    \vspace{20mm}
  \item 函数 $f, g$ 定义在 $\mathbb{R}$ 上且互逆,即对任意 $x, y \in \mathbb{R}$, 有 $g(f(x)) = x, f(g(y)) = y$ 成立。设 $f$ 可以写成线性函数与周期函数之和 $f(x) = kx + h(x)$, $k$ 为常数, $h(x)$ 为周期函数。求证:$g$ 也可以写成线性函数与周期函数之和。
  \vspace{60mm}
  \item 已知 $f(x), g(x)$ 均为 $\mathbb{R}$ 上的周期函数,且有 $|f(x) - g(x)| \leq (\frac{1}{2})^x$ 对 $x \in \mathbb{R}$ 恒成立,求证:$f(x) \equiv g(x)$。
  \vspace{100mm}
  \item 证明:函数 $f(x) = 2^x + 3^x + 9^x$ 不能表示为有限个周期函数的和。
\end{enumerate}
\vspace{40mm}

\newpage
\section*{奇偶性}
\subsection*{定义:}
\subsubsection*{\textbf{偶函数:}}
定义在$\mathbb{D}$上的函数$f(x)$,满足$\forall x \in \mathbb{D}$,都有$f(-x) = f(x)$,则$f(x)$为偶函数。\\
- \textbf{直观理解:} 图像关于y轴对称,定义域关于原点对称。

\subsubsection*{\textbf{奇函数:}}
定义在$\mathbb{D}$上的函数$f(x)$,满足$\forall x \in \mathbb{D}$,都有$f(-x) = -f(x)$,则$f(x)$为奇函数。\\
- \textbf{直观理解:} 图像关于原点对称,定义域关于原点对称。

\subsection*{例题:}
\begin{enumerate}
  \item 已知 $f(x)$ 是定义在 $\{x | x \neq 0, x \in \mathbb{R}\}$ 上的偶函数,当 $x > 0$ 时,$f(x) = x^2 - x + 1$,求 $f(x)$ 的解析式。
  \vspace{20mm}
  \item 已知 $f(x)$ 是定义在 $\{x | x \neq 0, x \in \mathbb{R}\}$ 上的奇函数,当 $x > 0$ 时,$f(x) = x^2 - x + 1$,求 $f(x)$ 的解析式。
  \vspace{20mm}
  \item 证明: 任何一个定义在 $\mathbb{R}$ 上的函数 $f(x)$ 都可以表示为一个奇函数和一个偶函数的和。
  \vspace{40mm}
  \item 已知 $f(x)$ 是定义在非零实数上的函数,$\forall x,y \in \mathbb{R}: f(xy) = f(x) + f(y) - 2$,求证: $f(x)$ 为偶函数。
\end{enumerate}

\newpage
\section*{单调性}
\subsection*{定义:}
\subsubsection*{\textbf{单调:}}
设函数 $f(x)$ 的定义域为 $\mathbb{D}$,区间 $I \subseteq \mathbb{D}$,若对所有 $x_1, x_2 \in I$,$x_1 < x_2$,都有 $f(x_1) \leq f(x_2)$,则称 $f(x)$ 在 $I$ 上是(单调)增函数。

设函数 $f(x)$ 的定义域为 $\mathbb{D}$,区间 $I \subseteq \mathbb{D}$,若对所有 $x_1, x_2 \in I$,$x_1 < x_2$,都有 $f(x_2) \leq f(x_1)$,则称 $f(x)$ 在 $I$ 上是(单调)减函数。

\subsubsection*{\textbf{严格单调:}}
设函数 $f(x)$ 的定义域为 $\mathbb{D}$,区间 $I \subseteq \mathbb{D}$,若对所有 $x_1, x_2 \in I$,$x_1 < x_2$,都有 $f(x_1) < f(x_2)$,则称 $f(x)$ 在 $I$ 上是严格(单调)增函数。

设函数 $f(x)$ 的定义域为 $\mathbb{D}$,区间 $I \subseteq \mathbb{D}$,若对所有 $x_1, x_2 \in I$,$x_1 < x_2$,都有 $f(x_2) < f(x_1)$,则称 $f(x)$ 在 $I$ 上是严格(单调)减函数。

\subsection*{例题:}
\begin{enumerate}
  \item 已知 $f(x) = x + \frac{a}{x}$ 在区间 $[2,+\infty)$ 上是严格增函数,求 $a$ 的范围?
  \vspace{40mm}
  \item 求函数 $f(x) = x^2 + \frac{1}{x}$ 的单调区间?
  \vspace{40mm}
  \item 已知 $f(x) = \sqrt{x^2 + 1} - ax$ 在 $x \in [0,+\infty)$ 是单调函数,求 $a$ 的范围?
  \vspace{40mm}
\end{enumerate}

\end{document}
