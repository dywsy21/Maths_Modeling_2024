\documentclass{article}

% Language setting
% Replace `english' with e.g. `spanish' to change the document language
\usepackage[english]{babel}

% Set page size and margins
% Replace `letterpaper' with `a4paper' for UK/EU standard size
\usepackage[letterpaper,top=2cm,bottom=2cm,left=3cm,right=3cm,marginparwidth=1.75cm]{geometry}

% Useful packages
\usepackage{amsmath}
\usepackage{graphicx}
\usepackage[colorlinks=true, allcolors=blue]{hyperref}

\title{优化农业种植策略以实现可持续发展}
\author{研究团队}

\begin{document}
\maketitle

\begin{abstract}
本文提出了一种数学模型,用于优化2024年至2030年间某村庄农业用地的种植策略。该模型旨在最大化经济收益,同时考虑土地可用性、作物需求和市场条件等约束。通过线性规划方法,本研究为提高农业生产力和经济可行性提供了一种可持续的方法。
\end{abstract}

\section{引言}

农业优化对于可持续发展至关重要,尤其是在土地资源有限的地区。本研究聚焦于华北山区的一个村庄,该地区气候限制了大部分农田每年只能种植一个季节。研究的目标是制定一种种植策略,在遵循土地类型适宜性和作物轮作要求的同时,最大化经济收益。

\section{方法}

本研究采用线性规划模型来优化种植策略。模型考虑了土地类型适宜性、作物轮作和市场条件等多种约束。利用2023年的数据,包括预期销售价格、产量和成本,来指导模型的构建。目标函数旨在通过考虑每种作物的预期销售量和相关成本来最大化利润。

\section{结果与讨论}

模型为2024年至2030年的种植策略提供了优化方案。结果显示,通过合理的种植规划,可以显著提高经济收益,同时满足各种约束条件。该策略不仅提高了农业生产力,还为村庄的可持续发展提供了支持。

\bibliographystyle{alpha}
\bibliography{sample}

\end{document}
